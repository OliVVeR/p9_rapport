It is important to have a reliable schedule that ensures communication is possible at expected times for systems operating in inaccessible areas, such as space, where communication is not always available.\\
This project aims to develop a tool, easing the process of making schedules as well as reducing the chance of human errors.
Said tool is made by utilising \acrshort{cora} and \acrshort{smc}, to produce a schedule, verify it, and finally output data relevant to the schedule, in one automated process.
The tool takes a payload and nanosatellite specification as input in order to construct a representative model made in \acrshort{cora}.
The result thereof is an optimal schedule that has the greatest return in profit, while still adhering to a set of restrictions, formalised as payload dependencies and battery considerations.
The schedule is verified using \acrshort{smc} that performs robustness checks in order to verify its stability.
It is important to perform these checks as there are many uncertainties that may dramatically influence the schedule when shipped into orbit.
\glsresetall
