Systems operating in inaccessible areas such as space, where communication is not always available, therefore its important to have a reliable schedule that ensures communication is possible at expected times. Errors occurring during these phases can have significant implications on a nanosatellite, if not handled correctly.\\
This project aims to develop a tool, easing the process of making schedules as well as reducing the chance of human errors. The development for this tool begins with an analysis of previous work done within the area of scheduling for nanosatellites, leading to a description of potential improvements along with a purposed solution. Said tool is made by utilising \acrshort{cora} and \acrshort{smc}, to produce a schedule, test it, and finally output data relevant to the schedule, in one automated process. This is solved by feeding data describing payloads and the nanosatellite environment to a model made in \acrshort{cora}, which produces an optimal schedule, based on profit. The script then produces a \acrshort{smc} model to perform robustness checks on the schedule, in order to verify its stability.
%along with a custom script that transforms \acrshort{cora} and \acrshort{smc} data into the final schedule along with statistical data for this.\\

%The development for this tool beings with an analysis of previous work done within the area of scheduling for nanosatellites, leading to a description of potential improvements along with a purposed solution. 
%This is solved by feeding data describing payloads and the nanosatellite environment to a model made in \acrshort{cora}, which produces an optimal schedule, based on profit. The script then produces a \acrshort{smc} model to perform robustness checks on the schedule, in order to verify its stability.
\glsresetall