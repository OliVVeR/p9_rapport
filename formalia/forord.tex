\chapter*{Preface}
This report documents the work of three \nth{9} semester Software Engineering students during the fall of 2017.
This report is as part of the specialisation in the field Semantics and Verification. The report describes the choices, and research associated with the implementation of a system for generating schedules for nanosatellites, as well as a description of relevant parts of said system.

The system has been developed using Python 3, written in PyCharm, and UPPAAL v4.0.2, and UPPAAL v4.0.19, and should only be run on Linux, for running the system we recommend it being done so on a 64bit machine. The models presented in \cref{sec:cora} \nameref{sec:cora} and \cref{sec:smc_model} \nameref{sec:smc_model} are made in the previously mentioned versions of UPPAAL and function as part of the implemented system. \\
Disclaimer: The included figures of said models represent all functionality of the actual models, but may look different as they have been made in Latex rather than using screenshots.\\

We would like to thank Lars Alminde for meeting with us, to tell us about GomSpace's work and allowing us to present our work. And also for looking over the data used for our implementation.

\section*{Reading Guidance}
The report is written in chronological order and should be read as such. Citations are made in accordance with the Vancouver style, meaning they are indicated by the use of numbers, and will look like: [1]. A complete list of cites is found in the bibliography and the cites are ordered by when they appear in the report. 
Citations at the end of a paragraph, to the right of the last period, references the entire paragraph, whereas cites written on the left side of a period references the given sentence.\\
Lastly, code found in listings may look different from what is found in the implemented code. No functionality has been changed, this is simply to make the code better presentable and easy to read.