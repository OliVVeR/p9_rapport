\chapter{Introduction} \label{cha:intro}
When creating a schedule for an embedded system that is sent into space, it is important that the schedule is correct according to the specified behaviour, as bad schedules may result in a depleted battery or other unwanted consequences. 
Some satellites are only able to communicate with their control centres on earth at given time intervals and will therefore need a schedule that lasts at least until the next communication opportunity.
There are many factors which need to be considered for such a schedule, such as time, power, and battery wear. 
Additionally, the harsh environment of outer space may affect the device and cause it to crash or otherwise hinder its execution.
This means the schedule has to be robust, as some unfavourable events may occur, such as a sudden need for a reboot which may cause the system to become delayed. 
If such an event occur it should be possible to correct for the delay.

This project aims to accommodate this problem by providing a tool that can produce schedules for such systems and test the robustness of the schedules in order to calculate how confident the tool is in its correctness.

\paragraph{Problem statement}
\textit{How is it possible to produce a schedule that consider multiple requirements, and how is it possible to verify the robustness of such a schedule?}

Generating a schedule with respect to profit and energy consumption has already been done with the GomX-3 mission\cite{gomx3}.
Their solution were to generate a schedule, that uses orbit and power predictions as input for the scheduler, which was modelled in UPPAAL \gls{cora}.\\
We have chosen their work as the starting point for our model, as we consider many of their findings relevant for solving the problem.

%our goal
Our goal for this project is to generate a validated schedule which is produced by analysing a model, which models the behaviour and battery of a nanosatellite.\\
We will try to achieve this by creating a model in UPPAAL \gls{cora} which will take a input configuration that represent the environment, and the payloads the user is trying to schedule.
The schedule will be tested in regards to its robustness by the use of UPPAAL \gls{smc}. %, a tool for statistical simulation and verification. 
When a schedule have been accepted it will be printed in a readable format, such that it can be inspected by the users.\\
The process should be automated such that the only necessary input from the user is the specified configuration.
The solution should be generalised so it may be used in other embedded systems and not just nanosatellites.
