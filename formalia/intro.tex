\chapter{Introduction}\label{cha:intro}
Scheduling in orbit can be a daunting task where many factors needs to be considered, this rapport aims to accommodate GomSpace in generation schedules for their nano satellites, GomSpace could benefit from having partial or fully generated schedules, and this rapport will go over how this could be achieved. GomSpace has a desired to alleviate the time required to create a schedule, this need has increased over the years since the amount of launched nano satellites has gone up. This trend has made it impractically to continue generating schedules by hands, and a more elegant solution is than needed.

Related work %Hvor bliver deres tasks defineret i GOMX-3? Hvorfor kan deres schedular ikke bruges til andre missioner? En af deres heuristics er king (6029126 states explored -> 175191 states explored)
Generation of a schedule with respect to profit and energy has already been done with the GomX-3 mission, they came up with an approach to generate a schedule, that uses orbit and power predictions as input for the schedule generator, which was modelled in UPPAAL CORA. One problem they countered was the large number of states when generation a schedule for 24 or more hours, to counter weight it they added heuristics that would reduce the number of options each state had in some cases. Two battery models was used, the first one is a simple model that was integrated with the schedule generator and the second one was an analytical model called kinetic battery model (KiBaM), which were used to validate candidate schedules. When a final schedule had been selected, a csv file will then be generated which has all the information about the schedule.

Problem statement
How can we design a tool that allows engineers to more easily generate a schedule that fits their needs, which can be configured and tweaked on a number of factors. to accommodate power level and other critical factors that are involved when operating a satellite in space.


%What did GomX-3 article do
%	Battery aware scheduling approach
%		TA modelling approach

%To maximize profit with a satellite, a schedule is needed.
%	Satellites schedule is often short, to better accommodate unforeseen event while maintain a certain profit level.
%	Making a schedule for a satellites can we extremely time consuming and error prone if done by hand.
%	Keeping track of power level with a schedule expands on the complexity.
%	To maximize profit the satellite should never go into safe mode (certain power level)
	


%%The solution should fit the problem statement	
%Solution
%	Four component goes into this tool
%	Number one, easy method of defining task and setting parameters given the specific satellite, this can be done in excel or other formats.
%	Number two, generate the best schedule.
%	Number three, validating that the schedule can actually run even if some of the task take a bit longer. If this can't be satisfied, the tool will tweak the parameter given in component one and generate a new best schedule until we can validate within a small margin of error.
%	Number four, exporting the results to the user in format that is useful to the engineers.

