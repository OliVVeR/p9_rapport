\chapter{Introduction}\label{cha:intro}
Scheduling in orbit can be a daunting task where many factors needs to be considered, shown in GomX-3 case, one of the most important factors were to monitor and maintain power. This rapport expands on GomX-3 case, which mainly focused on one satellite mission. We would like to see this approach widened to be suited for any space mission.


What did GomX-3 article do
	Generate schedule
	Verify schedule based on payload and power levels.

To maximize profit with a satellite, a schedule is needed.
	satellites schedule is often short, to better accommodate unforeseen event while maintain a certain profit level.
	making a schedule for a satellites can we extremely time consuming and error prone if done by hand.
	keeping track of power level with a schedule expands on the complexity.
	To maximize profit the satellite should never go into safe mode (certain power level)
	
Problem statement
	How can we design a tool that allows engineers to more easily generate a schedule that fits their needs, which can be configured and tweaked on a number of factors. to accommodate power level and other critical factors that are involved when operating a satellite in space.
	
Solution
	Four component goes into this tool
	Number one, easy method of defining task and setting parameters given the specific satellite, this can be done in excel or other formats.
	Number two, generate the best schedule.
	Number three, validating that the schedule can actually run even if some of the task take a bit longer. If this can't be satisfied, the tool will tweak the parameter given in component one and generate a new best schedule until we can validate within a small margin of error.
	Number four, exporting the results to the user in format that is useful to the engineers.
