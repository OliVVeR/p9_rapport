\chapter{Introduction}\label{cha:intro}
Scheduling in orbit can be a daunting task where many factors needs to be considered, this rapport aims to accommodate GomX in generation schedules for their nanosatellites. GomX could benefit from having partial or fully generated schedules, and this rapport will go over how this could be achieved. GomX has a desired to alleviate the time required to create a schedule, this has increased over the years since the amount of launched nanosatellites has gone up. This trend has made it impractically to continue generating schedules by hands.

%Finding an optimal schedule requires the machine to traverse a lot of paths to find the optimal solution, this often extensive use of memory and CPU.

What did GomX-3 article do
	Battery aware scheduling approach
		TA modelling approach

To maximize profit with a satellite, a schedule is needed.
	Satellites schedule is often short, to better accommodate unforeseen event while maintain a certain profit level.
	Making a schedule for a satellites can we extremely time consuming and error prone if done by hand.
	Keeping track of power level with a schedule expands on the complexity.
	To maximize profit the satellite should never go into safe mode (certain power level)
	
%The problem statement should be extracted from our introduction
Problem statement
	How can we design a tool that allows engineers to more easily generate a schedule that fits their needs, which can be configured and tweaked on a number of factors. to accommodate power level and other critical factors that are involved when operating a satellite in space.

%The solution should fit the problem statement	
Solution
	Four component goes into this tool
	Number one, easy method of defining task and setting parameters given the specific satellite, this can be done in excel or other formats.
	Number two, generate the best schedule.
	Number three, validating that the schedule can actually run even if some of the task take a bit longer. If this can't be satisfied, the tool will tweak the parameter given in component one and generate a new best schedule until we can validate within a small margin of error.
	Number four, exporting the results to the user in format that is useful to the engineers.


Related work
An article was published, 

