\section{Queries}\label{sec:queries}
We have created several queries in order to verify the schedule. We want to verity that we do not go below the battery threshold and the payloads that are schedules is executed. Additionally, we want to monitor some other variables in order to present some data for the user, such as the accumulated profit.

% impotant
\begin{equation} \label{eq:smc1}
	E<>\ o\_time == window\_max[n]\ \&\&\ ... dem\ der\ er\ dependant\ paa\ vinduet\ er\ koert
\end{equation}\afx{Ret den her når vi ved hvad der skal stå}
Query \ref{eq:smc1} is used to verify that the scheduled payloads that is dependant on a window, is executed within that window. Sudden restarts or variations in other timings may hinder the schedule in executing the payload within its window. 

% impotant
\begin{equation} \label{eq:smc2}
	Pr\ a < a*(threshold/100)
\end{equation}
Query \ref{eq:smc2} will result in a probability that reflects the risk of the battery going below the threshold. This threshold is specified in the configuration file from \cref{subsec:init} \nameref{subsec:init}. The nanosatellite will enter the safe mode if we go below the threshold.

It is expected that battery usage in the \gls{smc} model differs from the \gls{cora} as they use two  different battery models, Ideal and \gls{kibam}. The usage may also vary because of the possible difference in time usage that the individual payloads use. The time it takes to complete one payload in \gls{cora} are constant but it can vary in \gls{smc} as we want the model uncertainty.

We expect that we will use less energy in the \gls{smc} model because \gls{kibam} under estimates the energy consumption whereas the  \gls{cora} will overestimate. The \gls{smc} is also guaranteed to use the same amount, or less, of time on completing a payload as \gls{cora} will use. This is due to \gls{cora} using the worst case time when choosing the duration for payloads.

% impotant
\begin{equation} \label{eq:smc3}
	Pr\ counts == number\_of\_payloads har\ vi\ koert\ hele\ skidtemoeget?
\end{equation}\afx{Ret den her når vi ved hvad der skal stå}
\afx{beskriv query}

% extra
\begin{equation} \label{eq:smc4}
	simulate\ 1 [<=schedule\_time] \{ active\}
\end{equation}
Query \ref{eq:smc4} is used to output the order of payloads that was executed. This should correspond to the order that the trace dictates, but there could be differences if it was necessary to skip a payload.

% extra
\begin{equation} \label{eq:smc5}
	simulate\ 10 [<=schedule\_time] \{ a, b\}
\end{equation}
Query \ref{eq:smc5} provides two time series that reflects the two wells, a and b, from the battery model. This is useful for the user as they might want to discard the schedule if it uses more energy than what they comfortable with. Even if we do not go below the threshold, it may be to expensive to execute the schedule. % forklar hvorfor at vi kører 10 simulations?

% extra
\begin{equation} \label{eq:smc6}
	simulate\ 10 [<= schedule\_time] \{ profit \}
\end{equation}
Query \ref{eq:smc6} shows the accumulated profit for the schedule, and was introduced in \cref{subsec:csv} \nameref{subsec:csv}. We will not judge a schedule based on this value, we will let the user decide if it is acceptable. The profit is a abstract variable and we will not be able tell whether or not it is satisfactory.

\subsection{Parameter Tuning}