\section{Improvements and Additions}
In this section we will present the improvements we believe will be beneficial when producing a schedule, these will largely be based on \cite{gomx3}, trying to further develop the automation of schedules with added capabilities.

% Dynamic Scheduling
\cite{gomx3} wanted their schedule to be dynamic in the sense that the prices should be updated based on the schedule produces so far. This is a limitation in UPPAAL \gls {cora} that they wanted to sidestep, which they did by dividing their schedule into multiple disjointed subintervals. They would then carry over the state of the model from one subinterval to the next and then update the prices, based on the state. They noted that this was a trade-off between optimality and being dynamic because it made the schedule more greedy. We believe it is possible to avoid this solution of creating multiple schedules that has to be conjoined, while still being able to choose between payloads based on the state. %\afx{refernce til hvor vi har vores losning på dette. (vores brug af dpendencies)}

% robustness
The article therefore presented a solution for producing schedules that are specifically tailored to maximise payload utilisation while still minimising the risk of battery depletion. We believe it is possible to improve upon their solution by introducing the notion of robustness as there are uncertainties connected with how the satellite will behave and the environment it is operating in, because of the interference from the outside world.

% worthwhile/profit
The best way to preserve the battery capacity is to not execute any of the payloads, but this not a good idea as nothing useful is done, as pointed out in \textbf{Wognsen et al. 2015}\cite{score_function}. They identified that the battery lifetime would be extended if the nanosatellite would not execute any payloads that significantly changed the SoC, this observation is also true for this case even though we are just considering the battery capacity.\\
In order to insure that the nanosatellite is doing something worthwhile it is operating, we want to introduce the abstract \textit{profit} variable. A payload's profit signifies how much value it adds by executing it. Profit is not necessary expressed in currency, but could also reflect that experiment data have been collected or alike. This allows us to prioritise such that the more profitable(valuable) payloads are executed more often than the not so profitable.
%It does not help if the battery will last for 15 years if nothing useful is done in those years!

% battery wear
\cite{score_function} presented a battery lifetime scoring function that evaluated how much wear a SoC profile will have on a battery. One of their major points where that a great amount of wear can be avoided by performing shallow \gls{dod} cycling. It is desirable for nanosatellite to not wear the battery too quickly as it will reduce the total capacity, which may complicate the process of producing a schedule that will not drain the battery below the safety threshold. The article did also mention other aspects which may influence the battery lifetime such as dwelling at a certain SoC and the end of charge voltage which also affects the SoC. It would be beneficial to consider these aspects when producing the schedule, and finding a desirable ratio between payload utilisation and battery lifetime preservation.

The schedule we will produce will need to consider the requirements that was already made by the experience paper, staying above a SoC threshold, payload efficiency, payload windows etc. and the new ones we propose: monitoring of profit, reducing battery wear and being robust. These requirements results in the problem statement we presented in the introduction:\\
\textit{How is it possible to produce a schedule that considers multiple requirements, and how is it possible to verify the robustness of such a schedule?}\\
We will try and answer this by modelling the nanosatellite such that the model is capable of considering all of these requirements, and then producing a schedule which can be verified in regards to its robustness.