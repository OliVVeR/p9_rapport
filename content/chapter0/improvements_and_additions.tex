\section{Improvements and Additions}
In this section we will present the improvements we believe will be beneficial when producing a schedule, these will largely be based on the findings from Bisgaard et al. 2016\cite{gomx3}, trying to further develop the automation of schedules with added capabilities.

% Dynamic Scheduling
Bisgaard et al. 2016\cite{gomx3} wanted their schedule to be dynamic in the sense that the prices should be updated based on the schedule produced so far.
However, it is not possible to change the prices while a query is running in \gls{cora}.
They sidestepped this limitation by dividing their schedule into multiple disjointed subintervals. 
They would carry over the state of the model from one subinterval to the next and then update the prices based on the state.
All of the subintervals/subschedules would then be conjoined into a single schedule.
They noted that this was a trade-off between optimality and being dynamic because it made the schedule more greedy.
We believe it is possible to avoid this solution, while still being able to choose between payloads based on the state.

% robustness
A nanosatellite operating in space will be affected by the environment, such a Earth's magnetic field or the radioactive sun rays.
The schedule will therefore need to be robust, meaning being capable of handling errors, such as; the satellite will sometimes have to restart at unforeseen times, or maybe some dirt has covered parts of the solar panels which means less sunlight is absorbed and therefore the batteries will not be recharged as much as expected.
This may cause the satellite to fall behind schedule and it will therefore need to handle its new state, for example, some payloads may no longer be possible to execute as one more of their dependencies are not fulfilled. 
We believe that it is possible to improve upon the solution that Bisgaard et al. 2016\cite{gomx3} presented by introducing robustness, in order to deal with uncertainty.

% battery wear
Wognsen et al. 2015\cite{score_function} presented a battery lifetime scoring function that evaluated how much wear a \gls{soc} profile will have on a battery.
One of their major points were that a great amount of wear can be avoided by performing shallow \gls{dod} cycling.
\gls{dod} cycling refers to discharging to a certain \gls{soc} and the charging to 100\% again.
Shallow \gls{dod} cycling means that the discharges are relatively low such that the \gls{soc} stays above, for example, 75\% instead of fully discharging the battery.
It is desirable for the nanosatellite not to wear the battery out too quickly, as it will reduce the total battery capacity.
A battery with a low maximum capacity may complicate the process of producing a schedule that will not drain the battery below the safety threshold.
The article did also mention other aspects which may influence the battery lifetime such as dwelling at a certain \gls{soc} and the end of charge voltage which also affects the \gls{soc}.
It would be beneficial to consider these aspects when producing the schedule, and finding a desirable ratio between payload utilisation and battery lifetime preservation.

% worthwhile/profit
The best way to preserve the battery's capacity is to not execute any of the payloads, but this not a good idea as nothing useful is done, as pointed out by Wognsen et al. 2015\cite{score_function}. 
They identified that the battery lifetime would be extended if the nanosatellite would not execute any payloads that significantly changed the \gls{soc}, this observation is also true for this case even though we are just considering the battery capacity.\\
In order to ensure that the nanosatellite is doing something worthwhile, we want to introduce the abstract \textit{profit} variable. 
A payload's profit signifies how much value it adds by executing it. 
Profit is not necessary expressed in currency, it may reflect that payload have gathered data or that data have been transmitted back to Earth. 
This allows us to prioritise such that the more profitable(valuable) payloads are executed more often than the not so profitable.
This does not mean that the not so profitable payloads are never executed, but the scheduler will maximise the amount of profit gained over the duration of the schedule.

The scheduler will need to consider the requirements that was already specified by Bisgaard et al. 2016\cite{gomx3}, staying above a \gls{soc} threshold, payload efficiency, payload windows etc. and the new ones we propose: maximising profit, reducing battery wear and being robust. 
These requirements results in the problem statement we presented in the introduction:\\
\textit{How is it possible to produce a schedule that considers multiple requirements, and how is it possible to verify the robustness of such a schedule?}\\
We will try and answer this by modelling the nanosatellite such that the model is capable of considering all of these requirements, and then producing a schedule which can be verified in regards to its robustness.
