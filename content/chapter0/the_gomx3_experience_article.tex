\section{The GomX-3 Experience Article}
We introduced the GomX-3 article as the starting point for our project. They discuss the problem of resource management in regards to power consumption. They want to perform experiments and other missions when they are launching a nanosatellite and want to do so as efficient as possible. Efficient in the sense that they want to maximize their payload utilisation while in-orbit. 

% payload
A payload is a piece of equipment or software which helps the satellite achieve its goal. For example, the GomX-3 nanosatellite wants to use its transmitters or software defined radio module payloads. When we are referring to a payload we will refer to the usage of it, so when we are talking about the software radio it is the action of using it e.g. the process of transmitting or receiving data.

% Human error and models
GomSpace have several tools that assist them in producing a schedule for their nanosatellite, but some parts of the process is done by hand. This is a problem as it introduces a greater risk of human error in an environment where mistakes can be fatal for the missions. It is a highly complex task to manually plan the operations that the nanosatellite has to perform while still considering the battery capacity. This problem can be diminished by automating the process of scheduling the payloads, which they did do by modelling their nanosatellite in UPPAAL \gls{cora}. Their model allowed them to produce a schedule which their nanosatellite could interpret and execute.

% insolation
The GomX-3 nanosatellite is equipped with solar panels which it uses to power its hardware and to charge its battery which it draws upon when it enters eclipse, as the panels may only generate power when exposed to the sun(insolation). The battery's capacity is a resource that has be tightly monitored as all of the nanosatellite's operations consumes power and it will not be able to perform any functions if it is depleted.\cite{gomx3}

% constraints (windows and dependencies)
\cite{gomx3} made the observation that some payloads should be dependant on other payloads. If one payload represents the task of collecting experiment data, the payload that is responsible for downloading the data to earth may only be executed after the completion of the aforementioned payload. It would be wasteful to schedule a payload that downloads data if no data have been collected! Additionally, they observed that they would need to schedule two download payloads when one of their data collection payloads had been executed. This means that a simple dependency that checks whether or not a payload have been executed, is not enough. We will sometimes need to also check how many times the payload have been executed, and then later reset this counter in order to let this payload cycle restart.\\
The GomX-3 nanosatellite communicated with a specific centre one earth when it had to download its experiment data, but the centre was not always in line of sight and was therefore not able to always download its data. In order to download its data it must reside within an interval window which is defioned by two timestamps, one for when the payload becomes available and one for when it expires. \cite{gomx3} states that when a data collection payload is started, it must also finish within the window, as payloads that are aborted early or started late are not considered interesting.