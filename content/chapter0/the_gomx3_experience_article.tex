\section{The GomX-3 Experience Article} \label{sec:gomx3}
We introduced the GomX-3 article as the starting point for our project. They discuss the problem of resource management in regards to power consumption. They want to perform experiments with GomSpace on the GomX-3 nanosatellite, to collect insight on how to model and verify an efficient schedule for the nanosatellite. Efficient in the sense that they want to maximise their payload utilisation while in-orbit. 

% payload
A payload is a piece of equipment or software which helps the nanosatellite achieve its goal. For example, the GomX-3 nanosatellite wants to use its transmitters or software defined radio module payloads. When we are referring to a payload we will refer to the usage of it, so when we are talking about payload X which uses the radio module to transmit or receive data, we only care about the effect of that payload and not what physical modules that are required e.g. if the nanosatellite need to slew we only want to know how long it will take and not how it is done.

% Human error and models
GomSpace have several tools that assist them in producing a schedule for their nanosatellite, but some parts of the process is done by hand. This is a problem as it introduces a greater risk of human error in an environment where mistakes can be fatal for the missions. It is a highly complex task to manually plan the operations that the nanosatellite has to perform while still considering the battery capacity. This problem can be diminished by automating the process of scheduling the payloads, which Bisgaard et al. 2016\cite{gomx3} solved by modelling the GomX-3 nanosatellite in \gls{cora} by use of priced time automata. The model allowed them to produce a schedule which could be uploaded and execute on the nanosatellite.

% insolation
The GomX-3 nanosatellite is equipped with solar panels that it use to power its hardware and charge its battery which it draws upon during eclipse, as the panels may only generate power when exposed to the sun(insolation). The battery's capacity is a resource that has be tightly monitored as all of the nanosatellite's operations consumes power, and it will not be able to perform any functions if it is depleted.\cite{gomx3}

% power
The GomX-3 nanosatellite will enter a safe mode if the battery's \gls{soc} goes below a certain threshold. The safe mode prohibits the execution of all non vital processes, in order to maintain power until the battery goes above the threshold. Bisgaard et al. 2016\cite{gomx3} made it the primary objective of the schedule to avoid entering safe mode, because no payload can be execute during this mode.

% constraints (windows and dependencies)
Bisgaard et al. 2016\cite{gomx3} made the observation that some payloads should be dependant on other payloads. If one payload represents the task of collecting data, the payload that is responsible for transmitting the data to earth may only be executed after completing the aforementioned payload. It would be wasteful to schedule a payload that transmit data if no data have been collected! Additionally, they observed that they would need to schedule two transmitting payloads when one of their data collection payloads had been executed. This means that a simple dependency that checks whether or not a payload have been executed, is not enough. We will sometimes need to also check how many times the payload have been executed, and then later reset this counter in order to let this payload cycle restart.\\
The GomX-3 nanosatellite communicates with a specific centre one earth when it has to transmit its data, but the centre is not always in line of sight and it is therefore not able to always transmit its data. In order to transmit its data, it must reside within an interval window which is defined by two timestamps, one for when the payload becomes available and one for when it expires. Bisgaard et al. 2016\cite{gomx3} states that when a data collection payload is started, it must also finish within the window, as payloads that are aborted early or started late are not considered interesting.
