\section{Inaccuracies and Assumptions}\label{sec:in_and_ass}
During our development of the \gls{cora} and \gls{smc} models, we have made some decisions which may not always be an accurate representation of the outside world. In this section we will discuss each of these choices to evaluate the potential effect, these inaccuracies and assumptions may have on the produced schedule.

%More than one window for a payload \cref{ssec:multi_window}
In \myref{sec:read_input} we describe how a payload can have an associated window in which the payload can be executed, but the possibility of a payload having multiple windows was never explored, despite of our models supporting it. Unfortunately our payload descriptions semantics does not allow for multiple windows. This has the potential to make the produced schedule differ significantly from the actually optimal schedule, and we therefore believe this to be a significant inaccuracy.

Introducing the variable idle\_cost to our configuration file, see \myref{sec:read_input}, could reduce the computation time for producing a schedule in \gls{cora}. It is meant to reduce the number of payloads, by removing operational payloads that run frequently and adds no profit to the schedule and simulate their energy requirements via the idle\_cost variable. 
This could give misleading results since idle\_cost represent the average energy consumption for the operational payloads and continuously drain on the battery unlike payloads that only drains on the battery for short periods. However, by specifying the idle\_cost to be zero and describe all payloads in the payload description this variable will not cause any misrepresentation to the produced schedule.

In \myref{sec:cora} it was described that the user defined profit is used to calculate the cost in the \gls{cora} model. As \gls{cora} finds the best schedule in regards to the cost, profit works as a priority. Because of this it might be desirable for the user to define a wider range of priorities, especially in cases with many different payloads. However, we believe this to have a small effect on the produced schedule.

Given the restrictions of \gls{cora}, we modelled an Ideal battery model. This will according to the observations made during \myref{sec:kibam} overestimate the battery compared to a real battery, which may cause the produced schedules to consume more energy than what is available. The impact of this is mitigated with our \gls{smc} model having a more pessimistic battery model.

The Processor template in the \gls{cora} model, see \myref{sec:cora}, we mention two factors that can have an effect on the optimal schedule. A payload is always executed based on its worst execution time, and deadlines force the model to wait a period after the payload has been executed. Payloads running worst case have the benefit of taking more energy, and since the Ideal battery model overestimate the actual value by doing so is believed to get closer to the actual. 
Unfortunately schedules produced this way is not as payload efficient as they could be, especially because of the delay to wait for the deadline. However, maximising the payload efficiency i.e. discarding the wait after executing a payload, would leave the schedule more vulnerable to errors potentially resulting in a chain reaction of payloads being skipped based on the individual payloads dependencies and windows when the schedule is verified via \gls{smc}. We believe this to have a considerable impact on the schedule, but where exist a workaround for this by simply defining the deadline to equal the maximum execution time.\\
Additionally the way dependencies are modelled, does not allow for all forms of payload dependencies such as being dependant on "A" or "B", which can lead to less optimal schedules if the user is not able model their payload dependencies within our system. This can have a significant impact on the produces schedule. However, there is a form of workaround for this e.g. for the GomX-3 case, after having run one L-band it was needed to run two X-band. In our system the user can express the X-band as having a maximum runs of four and being dependant on one L1 and one L2 which each have a maximum runs of 1. This would result in a schedule starting with one of each L-band followed by four X-band. Because of this we believe this inaccuracy to be of minor importance. 

In \myref{sec:cora} when describing insolation we made two inaccuracies for the sake of performance. Dividing the number of updates made to the battery during an orbit, and insolation period lasting precisely half the orbit length. Determining the severity of reducing the number of updates to the battery during an orbit, showed a big decrease in time taking to produce the schedule with the effect of an added inaccuracy which we concluded to be negligible.\\
Assuming that orbits are constant is not realistic, however the only knowledge we have about the effect of the irregularity of an orbits is, that the inaccuracy increases over time. We therefore do not know the effect this will have on the schedule.

Modifications to the \gls{smc} model were also done, as described in \myref{sec:smc_model}, \gls{kibam} is one of the factors that impact the \gls{smc} model, unlike in \gls{cora} where the battery model overestimated the battery, \gls{kibam} underestimate resulting in few schedules being discarded even though the may be viable. We believe this to be a rare occurrence and will have little effect. 

%Starting value from future work \cref{ssec:start_val}
Our \gls{cora} model does not support knowledge from previous schedules, so the variable \uppVar{runs} which keep track of which payloads has been executed will always be zero initially, making it impossible to generate a schedule based on \uppVar{runs} from the previous schedule. This have a potential to greatly impact the schedule but also to be of no effect. 

To conclude, our system implements several inaccuracies most with little to no effect but others with a more significant impact. We believe the payloads with multiple windows and the missing start values for \uppVar{runs} to be significant and something that would require correction before actually using the system. The remaining we believe to have a small enough impact, either because of being negligible of because of the possible workarounds to continue using, with the exception of the perfect orbit as we simply do not know its impact.





