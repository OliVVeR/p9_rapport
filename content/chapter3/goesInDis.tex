\section{Evaluating inaccuracies and assumptions}
During our development of our \gls{cora} and \gls{smc} model, we have make some decisions that may not always have been in favour of realistic representing the state of the outside world. This section will go over each of choices and evaluate on the potential effect it will when all of the factor are taking into consideration.

In \cref{sec:cora} we mentioned the need for priorities to indicate the profit of each the payloads, the prioritisation is set to a value between one to five, five being the most important payload to run in terms of profit. This dose not immediately impose any inaccuracies unless the user defines more than five payloads each with their distinct priority from one another.
We used an Ideal battery model given \gls{cora} restrictions, This will according to observation made during \cref{kibam} overestimate the battery compared to a real battery, which produces schedules that uses more energy than it has. The impact of is mitigated with our \gls{smc} model.

\cref{sec:cora_pro} is about what status the current payload is in. Here we mentions two factor that can have an effect on the optimal schedule, payload always runs in worst running time, and deadlines force the model to wait a period after the payload has been executed. 

Payloads running worst case has the benefit of taken more energy from a battery model that already overestimate the actual value. But schedules produced will sometimes get less profit, compared to schedules generate based on base case for payloads, the drawback with that approach is the reduced energy taken from the battery and potentially skipping payloads due to some payloads taking a bit more time then the best case. This can cause a chain reaction of payloads being skipped based on the individual payloads dependencies and windows when the schedule is verified with \gls{smc}.

deadlines is only a problem if changes are made to the \gls{cora} model, this is because it is possible to set deadline for each payload to their worst case running time, negating any effect deadlines may have on the schedule. Deadlines are now only really usage to let payload execute in \gls{smc} model even when their starting time have passed, as long as it is within its own deadline.

In \cref{sec:cora_ins} describing insolation a few inaccuracies to reduce state space and overall running time for producing a schedule. Two mechanics where tweaked to increase performance, dividing the number of updates made to the battery during an orbit, and insolation period lasting precisely half the orbit length 

We made some estimation to determine the severity of reducing the number of updates to the battery during an orbit, that showed a big decrease in time taking to produce the schedule and an added inaccuracy which we concluded to be neglectable, another side effect that may occurred  if the orbit length divided with the number of updates produces a fractions, that will thus result in the \gls{cora} model not updating the correct amount of times for an orbit, due to \gls{cora} all decimals is rounded down the nearest integer.

taskWindow in \cref{sec:cora_tw} also have a minor inaccuracy in the form of orbits for the nanosatellite is circular and not oval, this is 







