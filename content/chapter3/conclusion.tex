\chapter{Conclusion} \label{sec:conclusion}
In this chapter we conclude on the report and project as a whole to evaluate the degree of which the problems presented during  \myref{cha:problem} have been fulfilled. This culminated in the following problem statement:

\enquote{How is it possible to produce a schedule that consider multiple requirements, and how is it possible to verify the robustness of such a schedule}

To conclude on the problem statement and to what degree the different requirements have been fulfilled, we must examine and evaluate each of the requirements, and the robustness.

As described in \myref{sec:cora} the \gls{soc} threshold is ensured by the \gls{cora} model, as exceeding this would result in a deadlock, causing the trace to be discarded. Furthermore, in the \gls{smc} model we precalculate the energy consumption to also enforce a payload can only be executed if it will not cause the \gls{soc} to fall below the threshold.  Which these percussions installed, \gls{soc} will always be kept above this threshold.\\
After having completed a payload our \gls{cora} model will immediately start execution of the next payload, if any payload is available. However, we do not currently execute payloads as soon as they are available if the model is idling, it does not check if a payload is executable until a change in the PaylaodWindow template occurs. Therefore the requirement for payload efficiency is only partially fulfilled.\\
If a payload is defined as only being able to execute in a window, it must never execute at any other time. 
This is ensured in both models, and the requirement for windows is therefore fulfilled \\
As the schedule is produced in \gls{cora} with the setting to find the best trace based on the defined profit, we have archived the goal of maximising profit.\\
Ways of preventing excessive wear on the battery is only considered in our \gls{smc} model by enforcing that when the battery reaches maximum capacity, it will stop recharging and only start again when a percentage of the capacity has been drained.
Given that more could have been done to capture the potential effect of battery wear this is considered partial done.\\
We described wanting to be able to consider, recharge fluctuations, sporadic restarts, and delay to the schedule in \myref{sec:imp_and_add}.
However, in the current build it is only possible to check the effects of delaying the schedule, and run some probability queries. This means it is not possible to guarantee the robustness to extend it was desired.

As discussed in \myref{sec:in_and_ass} there are several assumptions and inaccuracies associated with producing our schedule. Despite of us considering several of these to have little to no impact on the schedule, others will most definitely impact it. As this is the case, it will cause a different schedule than what is the actual optimal one.

In conclusion we are able to construct a schedule with the given parameters using \gls{cora}, and partially check its robustness through the \gls{smc} model, which give some indication for the confidence in the produced schedule. This achieve our goal in the main statement. However, evaluating some of the requirements to being partially done we will state that we partially achieve our objectives in generating an optimal schedule and doing some robustness checking on it.
