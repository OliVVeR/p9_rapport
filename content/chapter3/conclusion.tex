\section{Conclusion} \label{sec:conclusion}
In this section we conclude on the report and project to evaluate on the degree of which we fulfilled the problems discovered during \nameref{cha:problem} \cref{cha:problem} accumulating in the problem statement:

\enquote{How is it possible to produce a schedule that consider multiple requirements, and how is it possible to verify the robustness of such a schedule}

We successfully made a project which is capable of producing a schedule upholding the set of requirements presented in \cref{sec:imp_and_add}, this done using our \gls{cora} model along with the provided information about payloads and global settings. Furthermore we were able to check the robustness of the schedule utilising the \gls{smc} model to output statistical data about the schedule's performance. All of this is bundle into a complete solution to minimize or eliminate any knowledge of the tools used in our solution.

Through our requirements the following items need to be examined in order to determine the degree with have achieved our goals: \\
\textbf{\Gls{soc} threshold} - This is achieved, through our \gls{cora} model and \gls{smc} model, by deadlocking \gls{cora} traces going under specified \gls{soc} threshold. In our \gls{smc} model we calculate the expected cost of each payload before executing it, if we can not ensure that we stay above \gls{soc} threshold the payload will not be executed.\\
\textbf{Payload efficiency} - \\
\textbf{Payload windows} - \\
\textbf{Maximising profit} - \\
\textbf{Reducing battery wear} - Goal not reached \\
\textbf{Robustness} - We described being able to consider, recharge fluctuations, sporadic restarts, and delay on the schedule in \nameref{sec:imp_and_add} \cref{sec:imp_and_add}. But we are in the current build only able to check the effects of delaying the schedule, this is therefore only be partially achieved.



