\chapter{Discussion} \label{sec:discussion}
In this chapter we will discuss aspects of this project which could have been done differently whether it being a result of making a decision to exclude something, choosing one approach over another, or simplifications made to better the execution time of the system.

\subsection*{Iterative Adjustment of the Nanosatellite Configuration} \label{subsec:disc_itt}
Iterative adjustment of the nanosatellite configuration was never implemented, as we estimated that it was not possible to make a correct and non naive implementation within the time frame of this project.
The downside of not having this is that the user manually has to adjust their configuration and it may be difficult to decide which parts of the input that have to be tuned, as it can be difficult to identify what changes will have a noticeable impact. \\
The user is also limited in what they are able to change, as it is not possible to modify all of the internal variables, and some values should never be changed. Some are hidden anyway to lower the required knowledge of the internal tools, \gls{cora} and \gls{smc}, in order to make the system easier to use. 
%Instead of having to manually adjust the configuration, it could be possible to make the system self configurable at the end of an iteration. Either to explorer new schedules or a more aggressive robustness analysis. For example, \cref{fig:tool_act}  have added an additional step to the tool-chain by allowing the system to loop if the robustness queries were unsatisfied. The system would then have to adjust the variables, such as the [min, max]

Since the iterative adjustment of the nanosatellite configuration have become manual, we have supplied an updated version of the tool-chain which now handles a "No" and "Yes" equally in location 6.
\begin{figure}[h]
	\includegraphics[width=\textwidth]{graphics/flow_act.pdf}
	\caption{Flowchart that displays the current workflow of the system}
	\label{fig:tool_act}
\end{figure}

\subsection*{Battery Wear and Lifetime} \label{subsec:disc_life}
We did not implement any major features that would try to avoid straining the battery. 
But as Wognsen et al. 2015\cite{score_function} expresses, it is important to predict the battery wear for battery powered systems as it is a central part of predicting the system's lifetime.
Wognsen et al. 2015\cite{score_function} presented a scoring function to compare and evaluate the impact of battery usage profiles on cycle life. Their scoring system would have been simple to implement as the \gls{cora} model is always aware of the battery's \gls{soc}. However, the authors conclude that the score does not indicate to what effect the usage profile will have on the battery life. They do note that it is expected that with a large enough amount of experiments, it would be possible to tailor the function to a specific battery type, such as the one the GomX-3 nanosatellite is equipped with. \\
It is difficult to use the scoring system without knowing what the scores represent in actual wear, as it makes it hard to determine how significant getting a high score is compared to a low one. However, as we have a large focus on battery in this project, it would have been sensible to include battery ware

%We are currently not able to model that the battery deteriorate over time, but since the user can theoretically calculate the actual capacity of the battery and set the maximum capacity to that value, it is possible to pass the information to the model.
%However, a location have been added to the UPPAAL \gls{smc} model in order to avoid overcharging the battery, as explained in \myref{sec:smc_model}.\\
%We believe that since our tool has an emphasis on battery usage, the ability to model battery wear and avoid schedules that have a high impact on it, will be sensible to implement. 

\subsection*{Improving Insolation}
As seen in \cref{fig:cora_inso} in \myref{sec:cora} the initial location for the Insolation template is \uppVar{inSun}. This should be changed to allow schedules to start in the other location \uppVar{inEclipse}. Similar our clocks used in the template is always initially zero, this will create a problem for the user if they want to start their schedule at another time in the orbit. A solution could be to introduce a new initial committed location that has edges to \uppVar{inSun} and \uppVar{inEclipse}. On each of the new edges a guard will need to be placed, this should check the user defined value to indicate which starting location it should transition to. Additionally on both edges there should be an update, to set the clock to the correct time according to what the user has specified in the configuration specifications.

\subsection*{Zero Windows}
As of now the models, both \gls{cora} and \gls{smc}, does not support there being no defined windows in the payload description as several variables are dependant on this. However, as windows is an important concept for our context we do not believe this to be a problem, as at least one window will most likely always be defined. Also there is a workaround by defining all payloads as being dependant on the window $0-OrbitTime$, meaning that they are not restricted by the window.

\subsection*{Battery Importance}
During our meeting with Lars Alminde from GomSpace we discussed our approach as well as potential new approaches. He expressed that nanosatellites battery was not as important as initially assumed\cite{gom_space_conversation}. This may be a combination of multiple factors like increased battery capacity and more efficient solar panels. Given that our meeting with GomSpace was late in the project we decided to keep battery as an aspect of our problem statement.

\section{Evaluating inaccuracies and assumptions}
During our development of \gls{cora} and \gls{smc} model, we have made some decisions that may not always be in favour of realistically representing the state of the outside world. This section will go over each of the choices made to evaluate on the potential effect it can have when all of the factor are taking into consideration.

In \cref{sec:cora} we mention the need for priorities to indicate the profit for each of the payloads, the prioritisation can be set to a value between zero and five, five being the most important payload to execute in terms of profit. This dose not immediately impose any inaccuracies unless the user defines more than six payloads each with their distinct priority.
We used an Ideal battery model given \gls{cora} restrictions, This will according to the observations made during \cref{kibam} overestimate the battery compared to a real battery, which produces schedules that uses more energy than they have. The impact of this is mitigated with our \gls{smc} model having a more pessimistic battery model.

\gls{cora} processor \cref{ssec:cora_pro} is about what status the current payload is in. Here we mentions two factor that can have an effect on the optimal schedule. A payload is always executed based on its worst execution time, and deadlines force the model to wait a period after the payload has been executed. 

Payloads running worst case has the benefit of taken more energy from a battery model that already overestimate the actual value. But schedules produced this way may sometimes be less profitable, compared to schedules generate based on best case running time for payloads, the drawback with this approach is the reduced energy taken from the battery and potentially skipping payloads due to some payloads taking a bit more time then the best case. This can cause a chain reaction of payloads being skipped based on the individual payloads dependencies and windows when the schedule is verified with \gls{smc}. Additionally the way we modelled dependencies would not be able to represent all forms of payload dependencies this will only lead to less optimal schedules if user can not model their payload dependencies with our dependency rules.

Deadlines is only a problem if changes are made to the \gls{cora} model, this is because it is possible to set deadline for each payload to their worst case execution time, negating any effect deadlines may have on the schedule. Deadlines are intended to allow payloads in \gls{smc} model to be executed, even when their starting time have passed, as long as it is does not exceed the deadline defined by the payload being executed.

In \gls{cora} \cref{ssec:cora_ins} describing insolation a few inaccuracies to reduce state space and decrease running time for producing a schedule. Two mechanics where tweaked to increase performance, dividing the number of updates made to the battery during an orbit, and insolation period lasting precisely half the orbit length.

We made some estimation to determine the severity of reducing the number of updates to the battery during an orbit, that showed a big decrease in time taking to produce the schedule and an added inaccuracy which we concluded to be neglectable. Another side effect that may occurred  if the orbit length divided by the number of updates produces a fractions, resulting in \gls{cora} model not updating the correct amount of times for an orbit, due to \gls{cora}  rounding down decimals to the nearest natural number.

taskWindow in \cref{ssec:cora_tw} mentions that our orbit are modelled circular, where a more representative picture of and orbit would be more oval in shape. Unfortunately it is not possible check what consequences this may have.

Modifications to the \gls{smc} model were also been done, described in \cref{sec:smc_model}, \gls{kibam} is one of the factors that impact the \gls{smc} model, unlike in \gls{cora} where the battery model overestimated the battery, \gls{kibam} underestimate resulting in few schedules being discarded even though the may be viable. Additionally we still assume that charging of the battery occurs half and orbit length as well as orbit being circular. To represent the schedule in the same manner as the \gls{cora} model.

To conclude on this we can divide the behaviours into two categories, "will", and "may" effect schedule. will includes battery models, worst payload running time, general insolation because these factors will always have an effect on the produced schedule. May on the other hand can be tweaked by the user to have no effect on the produced schedule, these are number of updates per orbit on the battery, five prioritisation limit on payloads, deadline on payloads.
\jfx{Too short?? What can we then say?}








