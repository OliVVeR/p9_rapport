\section{Discussion} \label{sec:discussion}

\subsection{Iterative Adjustment of the Nanosatelite Configuration} \label{subsec:disc_itt}
Iterative adjustment of the nanosatelite configuration was never implemented. We decided that we it was not possible to make a correct and non naive implementation within the given time frame of this project. We have decided to include the feature as future work as it we believe it will make the system easier to use.

The downside of having the user to manually adjust their configuration is that they have to decide on which parts of the input that have to be tuned. An automatic process could be able to analyse the schedule for bottlenecks or other parts that would be candidates for change. It can be hard o manually find the variables that may lead to schedule that is not possible, or not robust. They are also limited in what the are able to change, as it is not possible to modify all of the internal variables. Some are hidden anyway to lower the required knowledge of the internal tools, UPPAAL \gls{cora} and \gls{smc}, in order to make the system easier to use. 
%Instead of having to manually adjust the configuration, it could be possible to make the system self configurable at the end of an iteration. Either to explorer new schedules or a more aggressive robustness analysis. For example, \cref{fig:tool_act}  have added an additional step to the tool-chain by allowing the system to loop if the robustness queries were unsatisfied. The system would then have to adjust the variables, such as the [min, max]

Since the iterative adjustment of the nanosatelite configuration have become manual, we have supplied an updated version of the tool-chain which now handles a "No" and "yes" equally in location 6.
\begin{figure}[h]
	\includegraphics[width=\textwidth]{graphics/flow_act.pdf}
	\caption{Flowchart that displays the actual workflow}
	\label{fig:tool_act}
\end{figure}


problem determine SOC when a battery ends 
logic or in dependencies

Battery Decay


\subsection{Proof of viability}
Vi har kun lavet stikproever
Det tilfldige aspekt goer det besvaergeligt (Random search)
