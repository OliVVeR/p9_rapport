\section{Future Work} \label{sec:future}
This section covers the ideas we believe is worth pursuing in the future.

\subsection{Memory Considerations}
This topic concerns the physical memory on the nanosatellite in respect to payloads that is producing some amount of data during execution. 
During our meeting with GomSpace, we were introduced to the problem of sending data cost efficiently, as the nanosatellite have multiple windows where it can send data back to earth. 
But depending on the geographical location, it would cost a fee to use other countries satellite dishes.
It would be interesting to prodcue schedules that would take this problem into consideration while also balancing power and payload utilisation.
This would make it so that some windows for the data sending payloads are more attractive than others because of their lower fee.

\subsection{Start Orbit Time}
When the user is specifying the configuration, it is necessary to specify the start SoC. Something similar for the would be useful for specifiyng where in the orbit the nanosatellite will start. Currently, we will always start at orbit time 0 but this might not reflect the nanosatellites actual position in the orbit.\\
A possible solution would be to allow the user to chose a number within the orbit length, which will be the start time for the nanosatellite. Alternatively, this number could be as an offset for the insolation/eclipse periods and the windows, such the that the start orbit time will still be at 0.

\subsection{Celestial Bodies Obstructing Line of Sight to Sun}
The most common celestial body to obstruct the nanosatellite from recharging is the moon with the exception of earth.
It is questionable how much this would affect the generations of schedules. If it were to have an real thread, the moon and nanosatellite would need to have close to similar orbital rotation, which is unlikely. \afx{kom ind paa at mak kune lave en procent for hvor laenge at insolation vs eclipse vare}

\subsection{Oval orbits}
Our models assumes that an orbit is circular so it take equal time the travel a half orbit from any starting point in the orbit. This was not include to simplify the model and deemed unnecessary because the generated schedules only spanning between 12 to 24 hours. 

\subsection{Payload Dependencies}
Currently a payload can state if another payload needs to be execute before allow in first payload to be executed. But no rule in our project define if a payload can be execute multiple times after the depending payload have finished running. In other words this mean that we only allow one execution before the depending payload are required again.

\subsection{Satellite Attitude \& Drag}
Our program have no notion of satellite attitude, this is intentionally left out, because a payloads execution is defined by the user and could just include the time it takes to slew the nanosatellite in the payload description. By doing it this way we reduce the number of possible schedules that can exist. Some complications arise when going this direction, if slew is calculated in a payload that mean if another payload needs to face the same angle it also needs to add the time required to slew the nanosatellite. which is wasteful because they would be able to save time if the could slew and perform two payloads and then resume its original orientation.

\subsection{Battery Decay}
Battery deteriorating over time is not directly supported in our solution, but since the user can theoretically calculate the actual capacity of the battery and use it as the input parameter for battery capacity, the can add that information to the model. Additionally battery decay is not something that our scheduler tries to minimize when generating a schedule. This decision was\afx{I need to write stuff here, please make it good.....}

\subsection{Schedule Shipping}
%4.3 Schedule Shipping
%In order to uplink a schedule to GomX–3, several comma separated files (.csv)
%are generated. Each file contains a list of job opportunities of a certain type,
%for example L-Band (see below), given by two timestamps representing the start
%time and the end time of the job window respectively, the implied duration of
%the timestamps, as well as a flag that shows whether the opportunity should be
%taken.

\subsection{Bandwidth}
%Læg op til at man se på båndbredde da det også er en ressource
%    lav tig når vi idler for at gøre dataen mindre
%        komprimering, fjern afstikkere, udregn facit 
