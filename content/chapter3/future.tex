\chapter{Future Work} \label{sec:future}
This chapter covers the ideas we believe is worth pursuing in the future, or was not pursued in this project due to the limited project period.

\subsection*{Memory Considerations}
This topic concerns the physical memory on the nanosatellite in respect to payloads that is producing some amount of data during execution. 
During our meeting with GomSpace, we were introduced to the problem of sending data cost efficiently, as the nanosatellite have multiple windows where it can send data back to Earth. 
But depending on the geographical location, it would cost a fee to use other countries satellite dishes.
It would be interesting to produce schedules that would take this problem into consideration while also balancing power and payload utilisation.
This would make it so that some windows for the data sending payloads are more attractive than others because of their lower fee\cite{gom_space_conversation}.
\paragraph*{Bandwidth}
%Læg op til at man se på båndbredde da det også er en ressource
%    lav tig når vi idler for at gøre dataen mindre
%        komprimering, fjern afstikkere, udregn facit 
Had we implemented the aspect of physical memory, we could also have considered bandwidth. This could be used to determine how long it would take send and receive data to and from Earth. Or instead of having a dependency between sending and collecting data, it would be possible to collect a curtain amount of data till the storage is full and to then send a sub part of the collected data if there is not enough time to send it all.

\subsection*{Start Orbit Time}
When the user is specifying the configuration, it is necessary to specify the start SoC. Something similar for this would be useful for specifying where in the orbit the nanosatellite will start. Currently, we will always start at orbit time zero but this might not reflect the nanosatellites actual position in the orbit.\\
A possible solution would be to allow the user to chose a number within the orbit length, which will be the starting time for the nanosatellite. Alternatively, this number could be an offset for the insolation/eclipse periods and the windows, such the that the start orbit time will still be at zero.

\subsection*{Celestial Bodies Obstructing Line of Sight to the Sun}
The most common celestial body to obstruct the nanosatellite from recharging is the moon with the exception of Earth.
It is questionable how much this would affect the generation of a schedule. If it were to have a real thread, the moon and nanosatellite would need to have close to similar orbital rotation, which is unlikely. We did not believe that this problem could effect schedule's quality, but it made us consider other similar problems.\\
We have made the assumption that the nanosatellite will spent an equal amount of time in insolation and eclipse. This will not always reflect the actual time distribution. We believe that this can be solved by introducing a constant in the configuration. The constant is a percentage that describes the time spent in insolation or eclipse. 

\subsection*{Oval Orbits}
Our models assumes that an orbit is circular so it take equal time the travel a half orbit from any starting point in the orbit. This was not include to simplify the model and deemed unnecessary because the generated schedules only spanning between 12 to 24 hours. 

\subsection*{Payload Dependencies}
Currently a payload can be dependant on the successful execution of another payload. We have introduced dependencies in order to make it possible for the user to help decide the order of the payloads, such that it is only possible to send data after it have been collected.\\
This feature can be improved upon as there is not always a one-to-one relation between the number of times a payload should be executed. In the GomX-3 experience paper\cite{gomx3}, they observed that one L-band payload would require two X-band payloads in order to send all of the data that had been collected. 
Our models does not directly allow for such a relation as the completion of the dependency unlocks the execution of the dependant payload until a function locks it again. Currently, the locking function is only called when all payloads are no longer allowed to be executed because they have hit their maximum of allowed executions. 
This is what we are referring to when we state that it is possible to indirectly make this relationship, as the maximum allowed executions can be set to the same amount as the dependant payload requires it to execute. By this approch it can be defined that \textit{L} may be run once, and \textit{X} is dependant on \textit{L} and may be run twice.\\
This however is more of a workaround and we believe that an actually implementation of these dependencies would increase the expressiveness of the payload descriptions

\subsection*{Satellite's Attitude \& Drag}
The attitude of the nanosatellite is not modelled, which results in the schedule not being as payload efficient as it could have been if we did consider the attitude. Bisgaard et al. 2016\cite{gomx3} declared that the X and L-band equipment is installed on opposite surfaces of the GomX-3 nanosatellite, which mean that only of them may be used at the time, as they have to be pointed towards Earth. They accommodated this by implementing logic in their model for slewing the nanosatellite into place before the payload was executed. Our approach is to advice the user into including the possible time it takes to slew the nanosatellite before a payload is executed. This has the effect that the worst case execution time of the payloads that are attitude dependent, is increased by that of time it takes to slew into place. Another effect is that the tool may produce a schedule were the nanosatellite has to slew back and forth multiple times in a row, instead of bundling payloads that needs the same orientation.\\
Support for setting the nanosatellite's attitude, and specifying the dependence of it on payloads would solve this inefficiency.
The positive effect of not considering this aspect is the reduced state space in the \gls{cora} model. However, We do believe that it is worth testing how a schedule may be produced differently, if the feature were to be implemented.
%Our tool have no notion of satellite attitude, this is intentionally left out, because a payloads execution is defined by the user and could just include the time it takes to slew the nanosatellite in the payload description. By doing it this way we reduce the number of possible schedules that can exist. Some complications arise when going this direction, if slew is calculated in a payload that mean if another payload needs to face the same angle it also needs to add the time required to slew the nanosatellite. which is wasteful because they would be able to save time if the could slew and perform two payloads and then resume its original orientation.


\subsection*{Battery Wear}
Battery deteriorating is not supported in our solution, but since the user can theoretically calculate the actual capacity of the battery and set the maximum capacity to that value, it is possible to pass the information to the model. Furthermore, the scheduler is not trying to minimise battery wear when producing a schedule. However, a location have been added to the \gls{smc} model in order to avoid overcharging the battery, as explained in \myref{sec:smc_model}.\\
Wognsen et al. 2015\cite{score_function} expresses how it is important to find a balance between not wearing the battery and doing something worthwhile with the nanosatellite i.e. executing payloads. We believe that since our tool has an emphasis on battery usage, the ability to model battery wear and avoid schedules that have a high impact on it, will be sensible to implement. 

\subsection*{Restarts}
In \myref{sec:imp_and_add} we described a desire for incorporating unpredictable restarts into the schedule, in order to test the schedules robustness. However, this was never implemented. Implementing this could have been done in several ways e.g. restarts could have been added to the schedule at random when generating the \gls{smc} model, or a septate template could have been added to the \gls{smc} model, which would cause the processor and scheduler templates to leave their current location for a duration of time.

\subsection*{Starting Values} \label{ssec:start_val}
In \myref{sec:in_and_ass} we discussed that, despite of the many input parameters fed to the system, we believe it could have been beneficial to include more in order to better represent the nanosatellites current state. Specificity we believe a representation of current amount of executions per payload should have been included. This is important as the last generated schedule may have ended with several payloads having been executed, potentially fulfilling some dependencies, meaning other payloads should be executable from the beginning of the new schedule.

\subsection*{Multiple Windows per Payload}\label{ssec:multi_window}
As described in \myref{sec:in_and_ass}, there are multiple stations on Earth with which the nanosatellite can communicate it would be a beneficial addition to allow a payload to be associated with multiple windows. As this is not currently included it may have a significant impact on the generated schedule, since the nanosatellite may be ready to transmit data to Earth at time 20 but is only defined to have a window at time 70-80 despite of it, in reality, also having one at time 30-40. This could be included by making an update to the script feeding the payload description to the models.

