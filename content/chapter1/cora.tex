\section{UPPAAL CORA} \label{sec:upp_cora}
\acrfull{cora} is a branch of UPPAAL, that uses linearly priced timed automata to find optimal paths satisfying certain goals, based on lowest accumulated cost\cite{cs_cora}. Due to the underlying structure of \gls{cora} it is only possible to do reachability checks, and does not allow for liveness or deadlocks checks. Lastly, \gls{cora} cannot guarantee termination unless the modelled system is acyclic and clocks are bound by invariants\cite{uppaal_cora_download}\cite{cora_tutorial}.

\gls{cora} can be configured in a set of different way depending on the selected options, one of these options is the \textit{best} diagnostic trace. 
Best mean that \gls{cora} ensures that the path found with this option selected is the best path with respect to minimising cost.
\Cref{fig:cora_eks} will be used in an example that illustrates how cost accumulate over time.

\begin{figure}[H]
	\centering
	\begin{tikzpicture}
	%Locations
	\node [init] (l0) [label={
		[align=left]above:
		\textcolor{name}{A}
	}, label={
		[align=left]left:
		\textcolor{invariant}{cost '== 1}\\
		\textcolor{invariant}{\&\& time <= 5}
	}] {};
	\node [location] (l1) [right of=l0, xshift=20mm, yshift=20mm, label={
		[align=left]above:
		\textcolor{name}{B1}
	}, label={
		[align=left]below:
		\textcolor{invariant}{cost '== 3}\\
		\textcolor{invariant}{\&\& time <= 10}
	}] {};
	\node [location] (l2) [right of=l0, xshift=20mm, yshift=-20mm, label={
		[align=left]above:
		\textcolor{name}{B2}
	}, label={
		[align=left]below:
		\textcolor{invariant}{cost '== 4}\\
		\textcolor{invariant}{\&\& time <= 10}
	}] {};
	\node [location] (l3) [right of=l2, xshift=20mm, yshift=20mm, label={
		[align=left]above:
		\textcolor{name}{C}
	}] {};
	%Edges
	\path[->,black, thick] (l0) edge node [midway, left][align=left]{
			\textcolor{guard}{time >= 3}} (l1);
	\path[->,black, thick] (l0) edge node [midway, left][align=left]{
			\textcolor{guard}{time >= 3}} (l2);
	\path[->,black, thick] (l1) edge node [midway, right][align=left]{
			\textcolor{guard}{time >= 10}} (l3);
	\path[->,black, thick] (l2) edge node [midway, right][align=left]{
			\textcolor{guard}{time >= 7}} (l3);
	\end{tikzpicture}
	\caption{Simple template made in \gls{cora}}
	\label{fig:cora_eks}
\end{figure}
From \cref{fig:cora_eks} we see a \gls{cora} template with four locations and four edges connecting them. 
Three of the locations have invariants bound to the clock \uppVar{time} and an associated cost rate. 
For every time unit that passed in the initial location \uppLoc{A}, the cost will increase by 1. 
The guard and invariant enforces that it will remain there for three to five units of time before transiting to one of the two neighbouring location, either \uppLoc{B1} or \uppLoc{B2}.\\ 
In \uppLoc{B1} the cost will increase with a rate of 3 per unit of time and will stay there until time reaches 10. 
Alternatively it may transition to \uppLoc{B2} where the cost rate is 4 and will stay there until time is between 7 and 10. 
After transitioning to one of these locations it will finally reach location \uppLoc{C}. \\
When \gls{cora} traverse the model it will produce a number of traces that reflects the possible choices it is able to make, choices such as for how long to wait before taking a transition and which one to take.
For example, it can wait for as long as \uppLoc{A} allows, and then can take the transition to \uppLoc{B1}, the cost of reaching \uppLoc{C} will then be 20 at time 10.
Another possibility is to once again wait 5 time units in \uppLoc{A} before taking the transition to \uppLoc{B2} before transitioning to \uppLoc{C} at time 7, the cost of reaching \uppLoc{C} will then be 13 at time 7, it will then be able to wait in \uppLoc{C} until time 10 with no additional cost.
\gls{cora} will now be able to discard the trace with the cost of 20 because it observed a better path.
\begin{align}
E<> Template.C
\label{eq:cora_get_c}
\end{align}
To get the optimal path to \uppLoc{C} we set diagnostic trace to best and run the query seen in \ref{eq:cora_get_c}, the query asks if \uppLoc{C} is reachable.
This will produce a path that is equal to best path from our example.
