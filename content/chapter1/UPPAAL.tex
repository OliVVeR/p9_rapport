\section{UPPAAL}
When planning satellite missions it is important, that it can be guaranteed no major or unforeseen errors will occur. This may be done via extensive testing or by using a modelling tool capable of verifying, that the proposed actions are correct\cite{cs_smc}. Several such tools exists e.g. Kronos, and UPPAAL. For this project we will be using UPPAAL, specifically the version \gls{cora}, as it have been presented as the more relevant solutions for this project especially after it being used for the GomX3\cite{gomx3}.

Common for all versions of UPPAAL, is that; it has global decelerations, and templates. One model may have several templates, each with its own local decelerations. The template itself consists of one to many locations and edges, edges connects the locations and when activated indicates a chance in state.\\
Edges may be decorated with; selects, guards, synchronizations, and updates. Selects are used for introducing new temporary variables. Guards are used to ensure that an edge is not activated prematurely. Synchronizations are used for activating multiple edges across templates simultaneously. Finally updates are used to chance variable values, and to call functions written in declarations.\\
Locations can be given a name, and an invariant. An invariant must always be evaluated to true e.g. if a location have the invariant $time <= 5$, at time five there will be a chance of state.\\
Also common for the versions of UPPAAL, is that queries can be written i order to ensure sustain properties are upheld, such as; is some location reachable, and will time ever exceed some amount. To complement this there is an available simulator allowing the user to monitor the behaviour of their model, in the simulator the user can decide step by step what the model does, or generate a random trace.


