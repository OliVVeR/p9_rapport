\section{Reading the Input} \label{sec:read_input}
In order to model the payloads and environment, we will need some information from the user.
The payloads are described through an CSV file where the dimensions represents the variables we will consider when producing the schedule.
The environment, the satellite's properties, and the robustness parameters are all described in an configuration file.\\
We have filled out both files to provide examples of how to use them. 
It is expected that the user changes all of the variables to represent their needs.

\subsection{CSV File} \label{subsec:csv}
The payloads are defined out from seven variables.
\begin{itemize}
	\item	Name
	\item	Time
	\item	Energy
	\item	Profit
	\item	Deadline
	\item	Dependencies
	\item	Window
\end{itemize}
An example of a CSV file with five payloads can be seen in \cref{lst:csv}.
The example payloads are based on those from the GomX-3 nanosatelite\ref{gomx3}.\\
The name should indicate what the payload represents and is only their for user as it will not be used in actual schedule.
A payload's name will be translated to a number when it has been modelled and when it is referenced in the schedule. 
The first payload defined in the CSV file will be number zero and the n'th payload will be named n-1.\\
The time dimension specifies how much time, in minutes, it will take to complete the payload.
It is defined by a range in order to allow uncertainty 
\begin{figure}[H]
\begin{lstlisting}[caption={An example of how five payloads can be defined}, label=lst:csv, language=text]
Name,	Time,	Energy,	Profit,	Deadline,	Dependencies,Window
Slew,	2-5,	10,		1,		-1,			-1,			-1
L1,		90-90,	20,		5,		150,		-1,			-1
L2,		90-90,	20,		5,		150,		-1,			-1
X,		15-15,	10,		30,		1,			L1|L2,		45-60
UHF,	15-15,	15,		1,		-1,			-1,			45-60
\end{lstlisting}
\end{figure}

\subsection{Ini File} \label{subsec:init}


\begin{figure}[H]
\begin{lstlisting}[caption={}, label=lst:ini, language=text]
ello
\end{lstlisting}
\end{figure}