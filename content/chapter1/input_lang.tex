\section{Reading the Input} \label{sec:read_input}
In order to model the payloads and environment, we will need some information from the user. The payloads are described through a CSV file where the dimensions represents the variables we will consider when producing the schedule. The environment, the satellites properties, and the robustness parameters are all described in a configuration file.\\
We have filled out both files to provide examples of how to use them. It is expected that the user changes all of the variables to represent their system.

\subsection{Payload Specification} \label{subsec:csv}
The payloads are defined by seven variables and are defined in a CSV file. An example of a CSV file with five payloads can be seen in \cref{lst:csv}. These payloads are based on those from the GomX-3 nanosatellite\cite{gomx3}.\\
The seven variables are name, time, energy, profit, deadline, dependencies, and window.\\
\textbf{Name} should indicate what the payload represents. When being modelled a payloads name will be translated into a number. The numeric name will also be used when it is referenced in the produced schedule. The first defined payload will be number zero and the n'th payload will be named n-1.\\
\textbf{time} specifies how much time, in minutes, it will take to complete the payload. It is defined by a range in order to allow for uncertainty in regards to the timing of the payloads. It is valid to specify a number that is higher than the orbit duration, but a time may not be below 0.\\
\textbf{Energy} specifies how high a load, in mAh, the payload applies on the battery every minute it is running.\\
%The total load over the duration of one payload can therefore be calculated by multiplying energy with time.
\textbf{Profit} is expressed within a range from 1 to 5, which signifies how valuable or profitable it is to complete a payload.
1 being the least profitable and 5 being the most.
In the example from below, L1 and L2 represents the use of the L-Band transmitter that the GomX-3 nanosatellite use to communicate with other satellites, which allows it to collect valuable data. Which is why it has been given a 5 in profit as it is the most profitable payload. The first payload Slew represents the action of slewing the satellite. This does not directly generate any profit for the satellite as no data is gathered. The accumulated profit will be monitored when producing a schedule making it possible to manually filter out schedules that results in a lower total profit over its duration.\\
\textbf{Deadline} is a positive integer and is used to cancel payloads which are delayed too much.
The minimum value is the maximum execution time time and the maximum is undefined.
The scheduler may chose to wait before executing a payload and we can therefore risk that it is no longer relevant to execute the next payload in the queue.
The decision for whether or not to cancel a payload in regards to its deadline goes as follows: \textit{if the time we have spent waiting for the next payload to start plus the time it takes to complete the payload, exceeds the deadline; then cancel}. We want make sure that we only start payloads that are guaranteed to finish within their deadline\\
%waited\_time + execute\_time > Deadline \longrightarrow cancel\\
\textbf{Dependencies} contains a list of other payloads which the current payload is waiting for to be completed. A payload may only be executed if all payloads expressed in its dependencies have previously been activated. The user may specify logic "or" to signify that just one of they payloads is needed to complete before the dependency is fulfilled.
%type a $|$ which means logic "or" or the can type $\&$ which means logic "and".\\
%The user can set parentheses to

Some payloads can only be executed in certain time \textbf{widows}, such as when they are above the communication station on earth.
This is why we have added the Window variable which restricts when the payload can be executed as it may only happen within the window. The window is specified by a range which goes from 0 to orbit time. They user may also not specify a window in which case it as allowed to be executed at any given time.

In addition we have added the concept of \textbf{MaxRuns} which indicates the amount of times each payload can be executed before the count is reset. This is in order to avoid running the same payload for the duration of the schedule as soon as its dependencies are fulfilled.
\begin{figure}[H]
\begin{lstlisting}[caption={An example of how five payloads can be defined}, label=lst:csv, language=text]
Name,	Time,	Energy,	Profit,	Deadline,	Dependencies,Window,MaxRuns
Slew,	2-5,	10,		1,		85,			-1,			-1,		3
L1,		90-90,	20,		5,		120,		-1,			-1,		3
L2,		90-90,	20,		5,		120,		-1,			-1,		2
X,		15-15,	10,		3,		30,			L1|L2,		45-60,	2
UHF,	15-15,	15,		1,		45,			-1,			20-65,	1
\end{lstlisting}
\end{figure}
\ofx{should we use an example that is actually supported? no L1|L2}
\subsection{Configuration Specification} \label{subsec:init}
The nanosatellite and its enviroment is specified by a configuration which consists of a list of variables and constants. The configuration is read from an INI file which we have provided an example of below in \cref{lst:ini}.
The configuration is divided into three sections System, Cora, and SMC.\\
The System specific variables describes the satellite and the time constraints.
The variables are:
\begin{itemize}
	\item schedule\_length
	\item orbit\_time
	\item battery\_capacity
	\item idle\_cost
	\item safe\_threshold
	\item soc
\end{itemize}
The \textbf{schedule\_length} defines how long the produced schedule should be, in minutes. \textbf{orbit\_time} defines how long it takes to complete one orbit, in minutes. The \textbf{battery\_capacity} is an constant that defines the maximum capacity of the on-board battery. We do not expect that the user defines every single operation that the satellite is capable of performing, which is why we have made an abstraction by introducing the \textbf{idle\_cost} variable. 
\textbf{idle\_cost} is the load that is imposes on the battery at all times, even when the satellite is idling, not currently executing any of the defined payloads.
This will allow us to disregard some of the payloads that the satellite may perform as they can be grouped together as one single background process which is always being executed.
The user should only consider payloads which can be executed in parallel to the defined payloads and only chose those which energy and time cost are trivial.
Failing to do so may result in a non representative model of their satellite and therefore a schedule where our guarantees may not be valid.\\
Based on the GOMX-3 article\cite{gomx3} we have decided to include the \textbf{safe\_threshold} variable.
This variable is used to determine when the nanosatellite's SoC is too low to continue executing the schedule.
If the nanosatellite go below this threshold, it is in danger of depleting the battery, which means it will not be able to communicate with the earth for a period of time.
This will result in a huge loss of potential profit as we are no longer able to execute profitable payloads and should therefore be avoided.
It is critical that the schedules we produce will never go below this threshold.\\
We have four variables that are related to the \gls{smc} model where three of them is used in its battery model.
\begin{itemize}
	\item rec\_rate
	\item f\_rate
	\item ac\_width
	\item certainty
\end{itemize}
\textbf{rec\_rate} defines the recharge rate when the satellite goes trough insolation, such that the value is added to the battery's capacity every minute.
\textbf{f\_rate} is a rate that is specific to the battery model we have chosen, and it is used to simulate the recovery effect of batteries.
The recovery effect will be described later in \cref{sec:kibam} \nameref{sec:kibam}, together with the \textbf{ac\_width} constant.
The constant \textbf{certainty} specifies how certain we reauire that UPPAAL \glossary{smc} should be in its results. UPPAAL \gls{smc} is described in \cref{sec:smc} \nameref{sec:smc}.
\begin{figure}[H]
\begin{lstlisting}[caption={An example of how the environment can be defined}, label=lst:ini, language=text]
[SYSTEM]
# System specific
# schedule length in minutes
schedule_length = 720
# orbit time in minutes
orbit_time = 90
# battery capacity in mAh
battery_capacity = 5400
# idle cost in mAh
idle_cost = 5
# safe mode threshold in percentage [0-100]
safe_threshold = 40
# The start SoC in mAh
soc = 5400

[CORA]
# Cora specific

[SMC]
# SMC specific
# Recharge rate
rec_rate = 0.2
# flow of available charge
f_rate = 0.0002324
# available width in relation to bound width
ac_width = 0.16667
# how certain should the system be in the results? Percentage [1-99]
certainty = 95
\end{lstlisting}
\end{figure}
At this point the \gls{cora} section does not contain any variables, the section is used as a placeholder for future versions of the system, as the model could be expanded to include variables specific to \gls{cora}. This will be touched upon in the \cref{sec:future} \nameref{sec:future}.

Now that we know what kind of payloads that the nanosatellite is capable of performing and we know about the environment it is in, we are able to start modelling the system.
