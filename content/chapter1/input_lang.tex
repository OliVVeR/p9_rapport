\section{Reading the Input} \label{sec:read_input}
In order to model the payloads, battery, and schedule properties, we will need some information from the user.
The payloads are described through an input file where the dimensions represents the variables which will be used when producing the schedule.
The other properties related to the nanosatellite, is described in a configuration file.\\
We have filled out both files to provide examples of how to use them. It is expected that the user sets all of the variables in order to represent their system.

\subsection{Payload Specification} \label{subsec:csv}
The payloads are defined by eight variables and are defined in a CSV file. An example of this file with five payloads can be seen in \cref{lst:csv}. 
These payloads are based on those from the GomX-3 nanosatellite\cite{gomx3}.\\
The eight variables are Name, Time, Energy, Profit, Deadline, Dependencies, Window, and MaxRuns.\\
\textbf{Name} should indicate what the payload represents. When modelled, a payloads name will be translated into a number in range 0 to N-1 where N is the amount of payloads. The numeric name will also be used when a payload is referenced in the produced schedule.\\
\textbf{Time} specifies how much time, in minutes, it will take to complete the payload. It is defined by a range in order to allow for uncertainty in regards to the timing of the payloads. It is valid to specify a number that is higher than the orbit duration, but a time may not be below 0.\\
\textbf{Energy} specifies how high a load, in mAh, the payload applies on the battery every minute it is being executed.\\
\textbf{Profit} is expressed within a range from 1 to 5, which signifies how valuable or profitable it is to complete a payload. 1 being the least profitable and 5 being the most. In the example from below, L1 and L2 represents the use of the L-Band transmitter that the GomX-3 nanosatellite use to communicate with other satellites, which allows it to collect valuable data. Which is why it has been given a profit of 5 as it is the most profitable payload. The first payload Slew represents the action of slewing the satellite. This does not directly generate any profit for the satellite as no data is gathered.\\
\textbf{Deadline} is a positive integer and is used to cancel payloads which are delayed too much. The minimum value for the deadline is that of the maximum execution time and the maximum is undefined. The scheduler may chose to wait before executing a payload and we can therefore risk that it is no longer relevant to execute the next payload in the queue.
The decision for whether or not to cancel a payload in regards to its deadline goes as follows: \textit{if the time we have spent waiting for the next payload to start plus the time it takes to complete the payload exceeds the deadline; then cancel}. We want to make sure that only payloads that are guaranteed to finish within their deadline start\\
\textbf{Dependencies} contains a list of other payloads which the current payload is waiting for to be completed. A payload may only be executed if all payloads expressed in its dependencies have previously been activated. The user may specify a dependency of multiple executions of another tasks. In the example, X is dependant on L1 and L2 to be completed twice, before it may self be executed.\\
Some payloads can only be executed in a certain time \textbf{Window}, such as when they are above the communication station on Earth.
This is why we have added the variable which restricts when payloads can be executed as it may only happen within the window. The window is specified by a range with the minimum value of 0 and the maximum is equal to the the orbit time. If a payload does not have an associated window it will be allowed to run at any given time, given its dependencies are fulfilled.\\
In addition we have added the concept of \textbf{MaxRuns} which indicates the amount of times each payload can be executed in a payload cycle. A payload cycle is completed when all of the payloads have been executed as many times as described by their MaxRuns value. Whenever a payload has been completed an associated counter, that belongs to the payload, is incremented. When all payloads have reached the MaxRuns value, their individual counter is set to $0$ and a new payload cycle begins. All dependencies are also being reset when a new cycle begins, which means that payloads will need to wait for their dependencies to be completed again. This is done to avoid executing the same payload for the duration of the schedule as soon as its dependencies are fulfilled. MaxRuns is a constant and all payloads must be set to $1$ or higher.
\begin{figure}[H]
\begin{lstlisting}[caption={Example of how five payloads can be defined}, label=lst:csv, language=text]
Name,	Time,	Energy,	Profit,	Deadline,	Dependencies,Window,MaxRuns
Slew,	2-5,	10,		1,		85,			-1,			-1,		3
L1,		90-90,	20,		5,		120,		-1,			-1,		3
L2,		90-90,	20,		5,		120,		-1,			-1,		2
X,		15-15,	10,		3,		30,			0 2 2 0 0,	45-60,	2
UHF,	15-15,	15,		1,		45,			0 0 0 1 0,	20-65,	1
\end{lstlisting}
\end{figure}

\subsection{Configuration Specification} \label{subsec:init}
The nanosatellite and its battery is specified by a configuration which consists of a list of variables and constants. The configuration is read from an INI file which we have provided an example of below in \cref{lst:ini}.
The configuration is divided into three sections System, CORA, and SMC.\\
The System specific variables describes the satellite and the time constraints.
The variables are:
\begin{itemize}
	\item schedule\_length
	\item orbit\_time
	\item battery\_capacity
	\item idle\_cost
	\item safe\_threshold
	\item soc
	\item rec\_rate
\end{itemize}
The \textbf{schedule\_length} defines how long the produced schedule should be, in minutes. 
\textbf{orbit\_time} defines how long it takes to complete one orbit, in minutes. 
The \textbf{battery\_capacity} is an constant that defines the maximum capacity of the battery. We do not expect that the user defines every single operation that the satellite is capable of performing, which is why we have made an abstraction by introducing the \textbf{idle\_cost} variable. 
\textbf{idle\_cost} is the load that is imposes on the battery at all times, even when the satellite is idling, not currently executing any of the defined payloads.
This will allow us to disregard some of the payloads that the satellite may perform as they can be grouped together as one single background process which is always being executed.
The user should only consider payloads which can be executed in parallel to the defined payloads and only chose those which energy and time cost are trivial.\afx{maybe beskriv i assumption. Man kan bare saette den til nul og skedulere alting selv}
Failing to do so may result in a non representative model of their satellite and therefore a schedule where our guarantees may not be valid.\\
Based on the findings from Bisgaard et al. 2016\cite{gomx3} we have decided to include the \textbf{safe\_threshold} variable.
This variable is used to determine when the nanosatellite's \gls{soc} is too low to continue executing the schedule.
If the nanosatellite go below this threshold, it is in danger of depleting the battery, which means it will not be able to communicate with Earth for a period of time.
This will result in a huge loss of potential profit as we are no longer able to execute profitable payloads and should therefore be avoided.It is critical that the schedules we produce will never go below this threshold.\\
\textbf{rec\_rate}  defines the recharge rate such that the value is added to the battery's capacity every minute that the satellite is in insolation.\\
We have three variables that are related to the \gls{smc} model where two of them is used in its battery model.\\
\begin{itemize}
	\item f\_rate
	\item ac\_width
	\item certainty
\end{itemize}
\textbf{f\_rate} is a rate that is specific to the battery model we have chosen, and it is used to simulate the recovery effect of batteries. The recovery effect will be described later in \cref{sec:kibam} \nameref{sec:kibam}, together with the \textbf{ac\_width} constant.
The constant \textbf{certainty} specifies how certain we require that \gls{smc} should be in its results. \gls{smc} is described in \cref{sec:smc} \nameref{sec:smc}.
\begin{figure}[H]
\begin{lstlisting}[caption={Example of how the environment can be defined}, label=lst:ini, language=text]
[SYSTEM]
# System specific
# schedule length in minutes
schedule_length = 720
# orbit time in minutes
orbit_time = 90
# battery capacity in mAh
battery_capacity = 5400
# idle cost in mAh
idle_cost = 1
# safe mode threshold in percentage [0-100]
safe_threshold = 40
# The start SoC in mAh
soc = 4800

[CORA]
# Cora specific

[SMC]
# SMC specific
# Recharge rate
rec_rate = 4.2
# flow of available charge
f_rate = 0.0002324
# available width in relation to bound width
ac_width = 0.16667
# how certain should the system be in the results? Percentage [1-99]
certainty = 95
\end{lstlisting}
\end{figure}
At this point the CORA section does not contain any variables, the section is used as a placeholder for future versions of the system, as the model could be expanded to include variables specific to \gls{cora}.

Now that payloads and nanosatellite is specified, we are able to start modelling the system.
